\documentclass[xetex]{beamer}

% Don't use this with XeTeX !
% \usepackage[utf8]{inputenc}
% \usepackage[T1]{fontenc}

\usepackage[frenchb]{babel}
\usepackage{lmodern}
% \usepackage{listings}

% Taken from old beamer presentation
% Some might be useless
% \usepackage{graphicx}
% \usepackage{enumerate}
% \usepackage{stmaryrd}
% \usepackage{wrapfig}
% \usepackage{amsmath}
% \usepackage{amsfonts}
% \usepackage{amssymb}


\title{Projet Réseaux de Kahn}
\author[Nguyen, Voizard]{Nguyen Le Thanh Dung \and Antoine Voizard}
\institute[ENS]{\'Ecole Normale Supérieure}

\usetheme{CambridgeUS}

\begin{document}

\begin{frame}
  \titlepage
\end{frame}

% \begin{frame}{Plan}
%   \tableofcontents
% \end{frame}

\section{Généralités}

\begin{frame}{Généralités}
  \begin{itemize}
  \item 4 implémentations distinctes :
    \begin{itemize}
    \item une avec des processus lourds / tubes
    \item 2 séquentielles
    \item une avec des sockets
    \end{itemize}
  \item Parfois inutilement compliquées
  \item Interface monadique $\Rightarrow$ on ne pouvait pas s'empêcher
    de s'inspirer un peu de Haskell !
    \begin{itemize}
    \item \texttt{let (\$) f x = f x}
    \item \ldots
    \end{itemize}
  \item L'implémentation séquentielle est la plus performante
  \end{itemize}
\end{frame}

\section{Forks/pipes}

\begin{frame}[c]
  \begin{center}
    \Huge \insertsection
  \end{center}
\end{frame}

\begin{frame}{Forks et pipes}
  \begin{itemize}
  \item Pour coder cette version, c'est simple :
  \item Prendre l'implémentation modèle avec les threads, puis remplacer
    \begin{itemize}
    \item \texttt{Thread.create} par des forks
    \item \texttt{Thread.join} par \texttt{Unix.waitpid}
    \item les files + mutex par des tubes
    \end{itemize}
  \item Tout marche ``tout seul''
    \begin{itemize}
    \item \texttt{Marshal} facilite la vie !
    \end{itemize}
  \end{itemize}
\end{frame}



\section{En séquentiel}

\begin{frame}[c]
  \begin{center}
    \Huge \insertsection
  \end{center}
\end{frame}
\begin{frame}{Séquentiel : vue d'ensemble}
  \begin{itemize}
  \item Utilisation du \emph{multitâche coopératif} pour simuler
    le parallélisme
  \item Faire en sorte que \texttt{bind}/\texttt{put}/\texttt{get}
    passent la main aux autres ``processus''
  \item Nécessite d'avoir du code qui coopère bien
  \item Un processus doit pouvoir capturer \emph{la suite de son exécution}
    pour la remettre à plus tard
    \begin{itemize}
    \item avec les continuations délimitées
    \end{itemize}
  \item 2 variantes de ces idées
  \end{itemize}
\end{frame}

\begin{frame}{Continuations}
  \begin{itemize}
  \item Une étape de calcul est une fonction qui reçoit en argument
    la suite
    \begin{itemize}
    \item \texttt{type ('a,'b) t = Cont of (('a -> 'b) -> 'b)}
    \item type des calculs capables d'accéder à leur continuation
    \end{itemize}
  \item fonctions monadiques \texttt{return} et \texttt{bind} pour
    séquencer des étapes en CPS
  \item pour exécuter le calcul : lui passer une continuation finale
  \item exemple : \texttt{Cont (fun k -> k)} interrompt tout et renvoie
    la suite du calcul
    \begin{itemize}
    \item c'est le mécanisme à la base de tout !
    \end{itemize}
  \item cf. \texttt{Control.Monad.Cont} en Haskell
  \end{itemize}
\end{frame}

\begin{frame}{Première variante}
  \begin{itemize}
  \item Adaptée de l'article \emph{A poor man's concurrency monad}
  \item Idée : processus = calcul dans une monade de continuation
    avec un type de retour particulier
  \item Le calcul fait une étape puis renvoie
    \begin{itemize}
    \item soit la suite du calcul
    \item soit ``stop''
    \item soit ``je veux lancer cette liste de processus en parallèle
      et les attendre''
      \begin{itemize}
      \item implémente \texttt{doco}
      \item différence avec l'article
      \end{itemize}
    \end{itemize}
  \end{itemize}
\end{frame}

\begin{frame}{Première variante (suite)}
  \begin{itemize}
  \item Une boucle se charge d'ordonnancer tous les processus
  \item (round-robin avec une file de processus en attente)
  \item Hack avec des ID de processus pour attendre que tous les processus
    lancés par un \texttt{doco} terminent
  \item Implémenté différemment que dans l'article
  \end{itemize}
\end{frame}

\begin{frame}{Deuxième variante : avec des coroutines}
  \begin{itemize}
  \item Résolution du problème en 2 étapes
    \begin{itemize}
    \item Coder une implémentation générale des coroutines à partir
      des continuations
    \item Les utiliser pour mettre en place le parallélisme
    \end{itemize}
  \item Implémentation de l'opérateur \texttt{yield} :
    \begin{itemize}
    \item \texttt{let yield x = Cont (fun k -> Yield (x,k))}
    \end{itemize}
  \item Réimplémenter \texttt{bind} pour séquencer 2 calculs
    en insérant un \texttt{yield ()} au milieu
  \end{itemize}
\end{frame}

\begin{frame}{Avec des coroutines (suite et fin)}
  \begin{itemize}
  \item Implémentation de \texttt{doco} = simple round-robin
  \item Chaque processus s'occupe d'ordonnancer ses fils
  \item Dans la 1ère variante il y avait 1 ordonnanceur global
  \item Dans tous les cas, les channels sont triviaux
  \end{itemize}
\end{frame}


\section{Réseau/sockets}

\begin{frame}[c]
  \begin{center}
    \Huge \insertsection
  \end{center}
\end{frame}

\begin{frame}{Architecture globale de l'implémentation réseau}
  \begin{itemize}
  \item Client/serveur
  \item Toutes les communications des clients passent par les serveurs
  \item Canaux identifiés par des numéros
    \begin{itemize}
    \item Une table de hachage côté serveur contient l'information des
      canaux en cours d'utilisation
    \end{itemize}
  \item Mini-protocole de communication :
    \begin{itemize}
    \item Requête = \texttt{(get|put)} + ID de canal + donnée sérialisée
    \end{itemize}
  \item Code fonctorisé, utilise des modules pour passer des paramètres de
    configuration
  \end{itemize}
\end{frame}

\begin{frame}{Côté client}
  \begin{itemize}
  \item Type des canaux : id + \texttt{(in|out)\_channel} Caml correspondant
    à un socket connecté au serveur
  \item \texttt{get} et \texttt{put} implémentés comme on pense
  \item \texttt{return} et \texttt{bind} implémentés comme avec les threads
  \item \texttt{doco} purement local ! (on y reviendra)
  \end{itemize}
\end{frame}

\begin{frame}{Côté serveur}
  \begin{itemize}
  \item Boucle pour accepter des clients
  \item 1 worker thread créé pour chaque connexion, traite les requêtes
    get/put qu'il reçoit
  \item Pour chaque canal, on a une FIFO contenant
    \begin{itemize}
    \item soit des données (excès de put)
    \item soit des demandes (excès de get), avec un callback\ldots
    \end{itemize}
  \item réception d'un put : appel callback ou ajout dans la queue
  \item réception d'un get : peut être bloquant pour le worker en question
  \end{itemize}
\end{frame}

\begin{frame}{Gros défaut}
  \begin{itemize}
  \item \texttt{doco} ne fait que créer de nouveaux threads sur le même client !
  \item on a l'infrastructure pour faire communiquer des clients
    sur différentes machines entre eux\ldots
  \item mais pas pour distribuer des tâches automatiquement
  \end{itemize}
\end{frame}

\begin{frame}{Ce que ça donne concrètement}
  \begin{itemize}
  \item On a un module qui implémente l'interface requise par le projet
    \begin{itemize}
    \item Il lance client et serveur sur la même machine
    \item Ils communiquent sur localhost
    \end{itemize}
  \item Pour avoir une application ``sur le réseau'', manipuler
    les fonctions spécifiques du module Socket
  \end{itemize}
\end{frame}

\begin{frame}{Contraste}
  \begin{itemize}
  \item Séquentiel : beaucoup d'efforts pour l'exécution
    concurrente, communication triviale
  \item Réseau : toute une infrastructure pour permettre la communication,
    recours à la ``solution de facilité'' pour la concurrence
  \end{itemize}
\end{frame}

\end{document}

