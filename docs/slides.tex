\documentclass[xetex]{beamer}

% Don't use this with XeTeX !
% \usepackage[utf8]{inputenc}
% \usepackage[T1]{fontenc}

\usepackage[frenchb]{babel}
\usepackage{lmodern}
% \usepackage{listings}

% Taken from old beamer presentation
% Some might be useless
% \usepackage{graphicx}
% \usepackage{enumerate}
% \usepackage{stmaryrd}
% \usepackage{wrapfig}
% \usepackage{amsmath}
% \usepackage{amsfonts}
% \usepackage{amssymb}


\title{Projet Réseaux de Kahn}
\author[Nguyen, Voizard]{Nguyen Le Thanh Dung \and Antoine Voizard}
\institute[ENS]{\'Ecole Normale Supérieure}

\usetheme{CambridgeUS}

\begin{document}

\begin{frame}
  \titlepage
\end{frame}

% \begin{frame}{Plan}
%   \tableofcontents
% \end{frame}

\section{Généralités}

\begin{frame}{Généralités}
  \begin{itemize}
  \item 4 implémentations distinctes :
    \begin{itemize}
    \item une avec des processus lourds / tubes
    \item 2 séquentielles
    \item une avec des sockets
    \end{itemize}
  \item Parfois inutilement compliquées
  \end{itemize}
\end{frame}

\section{Forks/pipes}

\begin{frame}{Forks et pipes}
  \begin{itemize}
  \item Pour coder cette version, c'est simple :
  \item Prendre l'implémentation modèle avec les threads, puis remplacer
    \begin{itemize}
    \item \texttt{Thread.create} par des forks
    \item \texttt{Thread.join} par \texttt{Unix.waitpid}
    \item les files + mutex par des tubes
    \end{itemize}
  \item Tout marche ``tout seul''
  \end{itemize}
\end{frame}

\section{Implémentations séquentielles}

\begin{frame}{Séquentiel : vue d'ensemble}
  
\end{frame}

\end{document}

