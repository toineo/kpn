\documentclass[a4paper, 11pt]{article}

% Locale/encoding : use XeTeX ! no fontenc/inputenc needed
\usepackage{fontspec}
\usepackage{lmodern}
\usepackage[frenchb]{babel}

\usepackage{fullpage}

% TeX Gyre Pagella = free Palatino clone
\setmainfont[Numbers=OldStyle]{TeX Gyre Pagella}


\title{Rapport de projet Systèmes et Réseaux : réseaux de Kahn}
\author{Nguyen Le Thanh Dung \and Antoine Voizard}

\begin{document}

\maketitle

\section*{Généralités}

Le projet a été codé en OCaml 4.00. (Le module Socket ne compile pas
sur les machines de l'école qui sont en version 3.12 !) Les cibles du
Makefile (qui utilise \texttt{ocamlbuild}) sont : 
\begin{itemize}
\item \texttt{example} (par défaut) : exemple simple affichant les
  entiers
\item \texttt{example\_sieve} : crible d'\'Eratosthène
\item \texttt{clean} : effacer les traces d'une ancienne compilation
\end{itemize}

Toutes les implémentations demandées ont été réalisées (il y a même 2
solutions séquentielles), même si celle pour le réseau est légèrement
insatisfaisante (cf. section dédiée). Nous avons donc généralement
atteint nos objectifs même si, le temps permettant, nous aurions aimé
avoir d'autres implémentations, faisant par exemple de la répartition
de charge.

Sur nos tests, les implémentations séquentielles se révèle la plus rapide,
sans doute parce qu'elle évite les coûts de synchronisation et de
communication, surtout que le \og vrai \fg parallélisme des processus lourds
est inutile, même sur une machine multicoeur, dans un exemple tel que
le crible où les données traversent le réseau de Kahn en série.

Remarquons que l'interface proposée permet d'écrire des programmes plus
généraux que ceux rentrant dans le cadre des réseaux de Kahn, et que
nos implémentations permettent généralement à ces programmes de
fonctionner. Ainsi, il est possible pour un processus d'écouter
simultanément sur plusieurs files de communications\ldots

% Paragraphe superflu ?
Nous sommes partis du template de projet basique fourni, dont nous
avons gardé le Makefile. Le fichier \texttt{kahn.ml} s'est retrouvé
divisé en 3 : fonctions utilitaires dans \texttt{Lib.ml},
implémentation avec des processus légers (que nous avons conservé presque à
l'identique) dans \texttt{Th.ml}, et interface dans \texttt{KPN.ml}.

Notons enfin la présence dans notre code d'opérateurs inspirés de
la librairie standard de Haskell, permettant parfois d'omettre des
noms de variables explicites (cf. \texttt{Operators.ml}).


\section*{Processus lourds et tubes (fichier \texttt{Pipe.ml})}

Il n'y a presque rien à dire au sujet de cette implémentation, en
grande partie reprise sur celle avec des threads. Les choix techniques
sont les plus évidents possibles : un processus par agent du réseau,
un tube pour chaque canal. La manipulation des tubes se faisant très
simplement en OCaml grâce aux in/out\_channel et au module Marshal,
il n'y a pas eu de difficulté ici.


\section*{Implémentations séquentielles (dossier \texttt{Seq/})}

Nous avons implémenté 2 solutions différentes pour simuler l'exécution
parallèle. Dans les deux cas, les canaux et les fonctions \texttt{put}
et \texttt{get} étaient triviaux (implémentés par de simples queues
sans avoir à se soucier d'exclusion mutuelle), tandis que toute la
difficulté résidait dans le fait de faire avancer des processus
concurrents. Pour cela, il fallait faire en sorte que le \texttt{bind}
force un processus à \og passer la main \fg aux autres. Il faut donc
écrire du code qui coopère bien ; si un des processus plante ou rentre
dans une boucle infinie, il en va de même pour tout le programme !

Les deux employant des continuations délimitées, une monade de 
continuations a été écrite dans \texttt{Continuation.ml} pour leur
utilisation commune.

L'implémentation dans \texttt{Proletarian.ml} est basée sur le papier [1]
cité dans le sujet du projet. Le framework proposé ne répondant pas
tout à fait à la spécification du projet, nous avons ajouté un nouveau
type d'action pour implémenter la fonction \texttt{doco}, et altéré
l'ordonnanceur round-robin proposé. Celui-ci utilise désormais une boucle
impérative avec une file FIFO mutable de processus, et attribue à
chaque processus un identifiant pour qu'un processus père puisse
attendre que ses fils se terminent.

L'autre implémentation repose sur des coroutines, mécanisme
traditionnel pour le multitasking coopératif. Celles-ci sont
implémentées grâce à la capacité d'une coroutine de capturer la
continuation jusqu'à la fin de la coroutine, et de la
renvoyer au parent qui pourra la rappeler. Une fois ceci en place, la
fonction \texttt{doco} n'a qu'à faire un round-robin sur la liste
des processus en arguments, qui font appel à \texttt{yield} quand ils
ont fini une unité de travail. (voir les commentaires du code pour
plus de détails)

Différence principale : dans le 1er cas, l'ordonnancement se fait
globalement avec une queue de tous les processus, alors que dans le
2ème cas, chaque processus est responsable de l'ordonnancement de ses
fils.

Une difficulté majeure aura été de bien comprendre le fonctionnement
concret des continuations et comment les exploiter au mieux. (Une
tentative d'implémenter les coroutines était d'utiliser des call/cc
imbriqués et des références, ça ne marchait pas bien !)


\section*{Communication par prises sur le réseau (dossier \texttt{Socket/})}

Le projet en réseau repose sur une architecture client/serveur. Un
serveur central s'occupe de faire transiter les messages entre
différents clients qui s'y sont connectés. Dès qu'un agent veut émettre/recevoir
une valeur, il communique au serveur l'identifiant du canal utilisé,
le type d'opération à effectuer et l'éventuelle valeur. Dans le cas
d'une lecture, le serveur renvoie la première valeur disponible, dès
qu'il y en a une (le client reste bloqué pendant ce temps).

Détaillons le fonctionnement des canaux. Côté client, chaque canal la
donnée d'un ID de canal et de in/out\_channel OCaml correspondant à un
socket pour communiquer avec le serveur. Ce dernier, de son côté,
maintient pour chaque canal une liste de soumissions en attente d'être
lues (put) ou de requêtes en cas d'absence de données (get), au moyen
d'une table de hachage contenant des files
(cf. \texttt{Socket/RequestObjectQueue.ml}). Les données sont
sérialisées sous formes de chaînes grâce au module Marshal et aux
fonctions d'échappement (note: \texttt{Scanf.unescaped} n'existe que
dans OCaml $\geq$ 4.00 !).

Le serveur utilise un thread par connexion (de client), et une requête
se charge, lorsque le serveur peut y répondre, de réveiller le thread
concerné (à l'aide d'une astuce avec des mutex) ; celui-ci pourra
alors répondre à son client (qui lui a fait une demande de lecture).

L'utilisation correcte des sockets constitue une bonne partie de la
difficulté ; en effet il y a des détails techniques et un petite
erreur peut empêcher le fonctionnement du tout.

Un module \texttt{Simple.ml} est fourni (et c'est vers celui-ci que
pointe \texttt{Kahn.Socket}) qui lance un client et un serveur sur
différents threads dans un même processus, communiquant sur
localhost. Ceci permet d'implémenter les fonctions de l'interface, et
de faire en sorte que les exemples fonctionnent sans modification de
code. Cependant, il est bien possible de faire tourner séparément
client et serveur, même sur différentes machines connectées par un
réseau (cf. l'exemple dans le dossier \texttt{Telegraph/}).

Reste quelques points de frustration. Le premier est les coûts des
communications forcées à passer par le serveur. Un modèle décentralisé
où une file de communication est une connection directe entre 2
clients (ce qui serait moins général mais suffirait à programmer des
réseaux de Kahn) serait plus efficace.

Le deuxième problème, et le plus gros, est que la fonction
\texttt{doco} est implémentée simplement comme une création de threads
dans le même client. On aurait apprécié de pouvoir lancer un client
sur chaque machine de l'ENS, et que la reconfiguration dynamique du
réseau distribue les nouveaux processus sur les machines inactives,
pour faire un calcul de grande ampleur ! Ici, faute de temps, ce n'est
pas le cas : s'il est possible de faire tourner différents processus
sur différentes machines, il faut le faire manuellement\ldots 


\end{document}
